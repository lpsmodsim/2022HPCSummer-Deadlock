\documentclass{article}

\usepackage{tikz} 

\usepackage{amsmath} % advanced math

% margins of 1 inch:
\setlength{\topmargin}{-.5in}
\setlength{\textheight}{9.5in}
\setlength{\oddsidemargin}{0in}
\setlength{\textwidth}{6.5in}

\usepackage{hyperref}
\hypersetup{
    colorlinks=true,
    linkcolor=blue,
    filecolor=magenta,      
    urlcolor=cyan,
    pdftitle={Overleaf Example},
    pdfpagemode=FullScreen,
    }

\begin{document}

    % https://stackoverflow.com/a/3408428/1164295
    \begin{minipage}[h]{\textwidth}
        \title{2022 Future Computing Summer Internship Project:\\(Title of project)}
        \author{Firstname Lastname\footnote{myemailaddress@domain.com}\ , 
        AnotherFirst AnotherLast\footnote{anemail@domain.com}}
        \date{\today}
            \maketitle
        \begin{abstract}
            This challenge problem was solved / progress was made
        \end{abstract}
    \end{minipage}

\ \\
% see https://en.wikipedia.org/wiki/George_H._Heilmeier#Heilmeier's_Catechism

%\maketitle

\section{Project Description} % what problem is being addressed? 

The challenge addressed by this work is ...

\section{Motivation} % Why does this work matter? Who cares? If you're successful, what difference does it make?

The users of this work include ...

\section{Prior work} % what does this build on?

See \cite{texbook} for prior work in this area.

\section{How to do the thing}

The software developed to respond to this challenge was run on (one laptop/a cluster of 100 nodes).

The software is available on (https://github.com/your/project/)

\section{Result} % conclusion/summary

The result is ...

\begin{thebibliography}{9}
\bibitem{texbook}
Donald E. Knuth (1986) \emph{The \TeX{} Book}, Addison-Wesley Professional.

\bibitem{lamport94}
Leslie Lamport (1994) \emph{\LaTeX: a document preparation system}, Addison
Wesley, Massachusetts, 2nd ed.
\end{thebibliography}

\end{document}
